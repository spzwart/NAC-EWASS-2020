\documentclass{article}
\usepackage[utf8]{inputenc}
\usepackage{multirow}
\usepackage{multicol}
\usepackage{pdfpages}

\title{NAC-EAS conference 2020 at Holiday Inn 1 to 3 July in Leiden}

\begin{document}

\maketitle

\section{Summary}

Leiden Observatory is pleased to welcome you to the 75th Netherlands
Astronomy Conference (NAC), which will take place as part of the
European Astronomical Society Annual Meeting from 29 June – 3 July
2020 at the Event \& Convention Center and Holiday Inn Leiden, The
Netherlands

The meeting is intended for professional astronomers to come together
to share their recent research results and meet science journalists,
industrial partners and science policy makers. An important part of
the NAC is the development of the astronomical {\em family} feeling.

We are expecting to present an exciting and varied program that ranges
from contributed talks by PhD students and their latest results from
the forefront of Dutch research, to a number of international, invited
lectures. We will organize special sessions on each of the three NOVA
networks (Galaxies; Stars and planetary systems; Compact
objects). 

In addition to the scientific sessions, the Astronomy Conference
program includes ample opportunity for networking over dinners, drinks
and social events. The venue is easy to reach by train or bicycle.

\section{Organizers}
\begin{center}
    \begin{tabular}{ccc}
        \textbf{Organizing committee}
        \hline
        Huub R\"uttgering (LOC/SOC)\\
        Simon Portegies Zwart (LOC/SOC)\\
        Jan Lub (LOC/SOC) \\
        Martijn Oei  (LOC) \\         
        Martijn Wilhelm (LOC) \\         
        Frits Sweijen          (LOC) \\         
        Lydia Stofanova (LOC) \\         
        Dirk van Dam  (LOC) \\         
        Rafa\"el Mostert  (LOC) \\         
    \end{tabular}
\end{center}


\section{Size of the audience: 200}
\section{Speakers}
\section{Poster session}


\begin{center}
    \begin{tabular}{ccc}
        \textbf{Organizing committee}
        \hline
        Huub R\"uttgering (LOC/SOC)\\
        Simon Portegies Zwart (LOC/SOC)\\
        Jan Lub (LOC/SOC) \\
        Martijn Oei  (LOC) \\         
        Martijn Wilhelm (LOC) \\         
        Frits Sweijen          (LOC) \\         
        Lydia Stofanova (LOC) \\         
        Dirk van Dam  (LOC) \\         
        Rafa\"el Mostert  (LOC) \\         
    \end{tabular}
\end{center}

\subsection{Venue}

We would like to have access to a one of the large rooms (75 people)
at the Holiday Inn, one small room (15 people) and an open space where
we can start our social-network building activities.


Multiple small to medium rooms are
available and one large main hall, for (probably) the plenary sessions
of EWASS. There is a vide available, which we may be able to claim as
the ``Dutch Garden'' with an activity such as a Lego sculpture to
build, a puzzle make or something else to keep people entertained
outside the talks, focussing on a Dutch theme within astronomy.

In total there are 12 meeting rooms and one large hall, the convention
center. The following pages show the rooms with their capacities in
various setups and the layout of all the rooms at a 1 to 250 scale,
respectively.


Request: 1 large auditorium (75-100 people), 1 small room (15 people) and attic.

\subsection{Posters}

Whereas EAS will be equipped with electronic posters, we during the
NAC would like to use paper posters. which will be exposed at the long
corridor at the venue.  In this way, the NAC posters remain distinct
from the EAS posters, and we can have our NAC posterprize as usual.

EWASS will not use physical posters, but present them on interactive
screens instead. For the NAC we will stick to physical paper
posters. These can be put up, for example, in the long hallway leading
to the large room, guiding everyone to walk past them.

\begin{center}
    \begin{tabular}{ccc}
        \textbf{Posterprize committee}
        \hline
        Frans Snik \\
        Lex Kaper \\
        Paul Groot \\
        Peter Barthel \\         
        SRON  \\         
    \end{tabular}
\end{center}

\subsection{Dinner}

We plan on organizing a separate Dutch NAC dinner separate from the
EAS dinner, which should be included in the conference fee.

\section{NAC 2020 Program}

We aim for four half days of NAC during the EWASS conference. These
will run from 13:00 to 17:00 each day. A range of topics will need to
be covered by both talks and ``hack'' sessions. This will require two
rooms for each half day; one for each.

\begin{tabular}{ll|l}
    & \multicolumn{2}{c}{\textbf{DAY STARTS}} \\
    \cline{2-3}
    \textbf{13:00 - 13:30} & \multicolumn{2}{c}{Vici talk} \\
    %\cline{2-3}
    \textbf{13:30 - 15:00} & PhD/PostDoc talks, topic 1 & Hack session topic 2 \\
    %\cline{2-3}
    \textbf{15:00 - 15:30} & \multicolumn{2}{c}{Coffee break} \\
    %\cline{2-3}
    \textbf{15:30 - 16:30} & PhD/PostDoc talks, topic 3 & Hack session topic 4 \\
    %\cline{2-3}
    \textbf{16:30 - 17:00} & \multicolumn{2}{c}{Spinoza talk} \\
    \cline{2-3}
    & \multicolumn{2}{c}{\textbf{DAY ENDS}}
\end{tabular}\\%

\subsection{Talks [auditorium]}

Each day will start with a focussed ``tough talk'', given by an
invited speaker who has won a Vici grant. Afterwards, there will be 2
hours and 30 minutes available for student talks, of 20 minutes each
(this leaves 10 minutes of room for switching speakers, technical
difficulties etc.). The day is closed by another invited talk, this
time by someone who has won a Spinoza laureat to give a lighter talk,
giving a broader overview of the science in that field.

The talks will cover a wide range of topics. Each session before and
after the coffee break will have its focus on one. Below is a proposed
list of topics:
\begin{multicols}{2}
    \begin{enumerate}
        \item Outreach
        \item Instrumentation
        \item Simulations and modeling
        \item Multi-messenger astronomy
        \item Imaging
        \item Spectroscopy
        \item Theory
        \item Inteferometry
    \end{enumerate}
\end{multicols}

\subsection{Hack sessions [Small room]}

During the talks, there will be more hands-on ``hack'' sessions in
parallel. Here people can come together to, for example, go to a
tutorial, a lecture or in another way more interactive session to
learn about particular topics, such as outreach or data reduction. A
list of ideas for these topics is summarized below. Some of these
session could be more hands-on, with participants actually doing work,
while others that are more or too involved to do no the spot
(e.g. LOFAR, GAIA), could perhaps be more like a talk/discussion
session to give more insight in e.g. how to obtain and reduce the
data.

\begin{multicols}{2}
    \begin{enumerate}
        \item Training on outreach
        \item AMUSE data reduction
        \item Python in astronomy
        \item Machine learning
        \item LOFAR data reduction
        \item NOVA-6
        \item COSMOSIM
        \item GAIA
    \end{enumerate}
\end{multicols}

\subsection{Social activity [Attic]}

Frans Snik

\section{Speakers}

Each session will need at least two invited speakers: one Vici grant
and one Spinoza laureat. This section will list possible speakers, to
invite, sorted by applicable topics. Institutes to consider are:

\begin{multicols}{2}
    \begin{enumerate}
        \item University of Groningen (RuG)
        \item Leiden University (LU)
        \item Radbout University (RU)
        \item VU Amsterdam (VU)
        \item ASTRON
        \item SRON
        \item NIKHEF
        \item contribution from Belgian astronomy
    \end{enumerate}
\end{multicols}

\subsection{Outreach}
\begin{enumerate}
    \item Pedro Russo (LU)
    \item Peter Barthel (RuG)
    \item Joeri van Leeuwen (UvA, Vici)
    \item Bas Haring
    \item Ionica Smeets
\end{enumerate}

\subsection{Instrumentation}
\begin{enumerate}
    \item Paul Groot (RU)
    \item Peter Jonker (RU)
    \item Jelle Kaastra (SRON)
    \item Hiroki Akamatsu (SRON)
\end{enumerate}

\subsection{Simulations and modeling}
\begin{enumerate}
    \item Adrian Hamers
    \item Amina Helmi (RuG, Vici + Spinoza)
\end{enumerate}

\subsection{Multi-messenger Astronomy}
\begin{enumerate}
    \item Gijs Nelemans (RU, Vici)
    \item Michiel van der Klis (UvA, Spinoza)
    \item Heino Falcke (RU, Spinoza)
\end{enumerate}

\subsection{Imaging}
\begin{enumerate}
    \item Marc Verheijen (RuG, Vici)
    \item Ewine van Dishoeck (LU, Spinoza)
\end{enumerate}

\subsection{Spectroscopy}
\begin{enumerate}
    \item Ignas Snellen (LU, Vici)
    \item Xander Tielens (LU, Spinoza)
\end{enumerate}

\subsection{Theory}
\begin{enumerate}
    \item Sera Markoff (UvA, Vici)
    \item Erik Verlinde (UvA)
    \item Chris van den Broek (NIKHEF, Vici)
\end{enumerate}

\subsection{Interferometry}
\begin{enumerate}
    \item Heino Falkce (RU, Spinoza)
\end{enumerate}

\subsection{Newly hired researchers}
\begin{enumerate}
    \item Aurora Simionescu (LU)
    \item Yamilia Miguel (LU)
    \item Elisa Constantini (UvA)
    \item Elena Sellentin (LU)
\end{enumerate}
\end{document}
