\documentclass{article}
\usepackage[utf8]{inputenc}
\usepackage{multirow}
\usepackage{multicol}
\usepackage{pdfpages}

\title{NAC-EAS conference 2020 at Holiday Inn 1 to 3 July in Leiden}

\begin{document}

\maketitle

\section{Summary}

Leiden Observatory hosts the 75th Netherlandss Astronomen Conferentie
(NAC), which will take place as part of the European Astronomical
Society Annual Meeting from 29 June – 3 July 2020 at the Event
\& Convention Center and Holiday Inn Leiden, The Netherlands

The meeting is intended for professional astronomers to come together
to share their recent research results and meet science journalists,
industrial partners and science policy makers. An important part of
the NAC is the development of the Ductch astronomical {\em family}
feeling.

We are expecting to present an exciting and varied program that ranges
from contributed talks by PhD students and their latest results from
the forefront of Dutch research, to a number of international, invited
lectures. We will organize special sessions on each of the three NOVA
networks (Galaxies; Stars and planetary systems; Compact objects).

In addition to the scientific sessions, the Astronomy Conference
program includes ample opportunity for networking over dinners, drinks
and social events. The venue is easy to reach by train or bicycle.

\section{Organizers}

    \begin{tabular}{l}
        \textbf{Organizing committee}\\
        Huub R\"uttgering (LOC/SOC)\\
        Simon Portegies Zwart (LOC/SOC)\\
        Jan Lub (LOC/SOC) \\
        Martijn Oei  (LOC) \\         
        Martijn Wilhelm (LOC) \\         
        Frits Sweijen          (LOC) \\         
        Lydia Stofanova (LOC) \\         
        Dirk van Dam  (LOC) \\         
        Rafa\"el Mostert  (LOC) \\         
    \end{tabular}


\subsection{Size of the audience: 200}

\subsection{Venue}

Request: 1 large auditorium (75-100 people), 1 small room (15 people) and the Vide (attic).

These should be available from Tuesday to Friday (4 days) from 13:00 to 17:00

The main talks will be in the auditorium. In the small room we will
organize splinter meetings and in the Vide we'll have the Dutch
garden.

In addition, we would like to use the hall-way from the hotel to the
main auditorium for posters.

\subsection{Social Dinner}

We plan on organizing a separate Dutch NAC dinner separate from the
EAS dinner, which should be included in the conference fee.

\subsection{Poster session}

Whereas EAS will be equipped with electronic posters, we, during the
NAC, would like to use paper posters. which will be exposed at the
long corridor connecting the hotel with the main venue.  In this way,
the NAC posters remain distinct from the EAS posters, and we can have
our NAC posterprize as usual.

    \begin{tabular}{l}
        \textbf{Posterprize committee}\\
        Peter Barthel (RUG) \\         
        Stephanie Cazeau (TU Delft)  \\         
        Paul Groot (RU) \\
        Lex Kaper (UvA) \\
        Frans Snik (LU) \\
    \end{tabular}

\subsection{The Dutch garden: Social-network building activities [Franks Snik]}

There is a vide available, which we may be able to claim as the
``Dutch Garden'' with an activity such as a Lego sculpture to build, a
puzzle make or something else to keep people entertained outside the
talks, focussing on a Dutch theme within astronomy.

\section{Speakers}

\section{NAC 2020 Program}

We aim for four half days of NAC during the EAS conference. These
will run from 13:00 to 17:00 each day. A range of topics will need to
be covered by both talks and hands-on sessions (in parallel).

\begin{tabular}{ll|l}
    & \multicolumn{2}{c}{\textbf{DAY STARTS}} \\
    \cline{2-3}
    \textbf{13:00 - 13:30} & \multicolumn{2}{c}{invited focus talk} \\
    %\cline{2-3}
    \textbf{13:30 - 15:00} & contributed talks & Hands-on session \\
    %\cline{2-3}
    \textbf{15:00 - 15:30} & \multicolumn{2}{c}{Coffee break} \\
    %\cline{2-3}
    \textbf{15:30 - 16:30} & contributed & hands-on session \\
    %\cline{2-3}
    \textbf{16:30 - 17:00} & \multicolumn{2}{c}{invited overfiew talk} \\
    \cline{2-3}
    & \multicolumn{2}{c}{\textbf{DAY ENDS}}
\end{tabular}\\%

\subsection{Talks [auditorium]}

The day starts at 13:00, after lunch and ends at 17:00.

Each day will start with a ``focus talk'', given by an invited speaker.

This presentation is followd by 2:30 hours of contributed talks (20 min each).
with 30 minutes coffee in between 

The day is closed by another invited talk. This second invitee should present an topic's overview.

The list of topics should include
\begin{multicols}{2}
    \begin{enumerate}
        \item Outreach (session \#1a)
        \item Instrumentation (session \#1b)
        \item Simulations and modeling (session \#2a)
        \item Multi-messenger astronomy (session \#2b)
        \item Imaging (session \#3a)
        \item Spectroscopy (session \#3b)
        \item Theory (session \#4a)
        \item Inteferometry (session \#4b)
    \end{enumerate}
\end{multicols}

\subsection{Hands-on sessions [Small room]}

During the talks, there will be more hands-on ``hack'' sessions in
parallel. Here people can come together to, for example, go to a
tutorial, a lecture or in another way more interactive session to
learn about particular topics, such as outreach or data reduction. A
list of ideas for these topics is summarized below. Some of these
session could be more hands-on, with participants actually doing work,
while others that are more or too involved to do no the spot
(e.g. LOFAR, GAIA), could perhaps be more like a talk/discussion
session to give more insight in e.g. how to obtain and reduce the
data.

\begin{multicols}{2}
    \begin{enumerate}
        \item LOFAR data reduction  (parallel with session \#1a)
        \item Training on outreach  (session \#1b)
        \item Machine learning  (session \#2a)
        \item NOVA-6  (session \#2b)
        \item COSMOSIM  (session \#3a)
        \item Python in astronomy  (session \#3b)
        \item GAIA  (session \#4a)
        \item AMUSE data reduction  (session \#4b)
    \end{enumerate}
\end{multicols}

\section{Involving the press [Marieke Baan]}

We would like to involve the press a bit more actively than during
earlier NACs.  At the moment we are in discussion with Marieke Baan on
organizing writing sessions in te press room by coupling a Dutch
astronomer with a journalist and have them write, together, an article
for the `{\em Kids Week} or a chapter of a children book on astronomy.

\section{Speakers}

We strive a a balance in senior an djunior researchers, in gender
balance, on topica coverage and geographical coverage.

We expect to inite people from the following institutions: University
of Groningen (RuG), Leiden University (LU), Radbout University (RU),
VU Amsterdam (VU), ASTRON, SRON, NIKHEF, contribution from Belgian
astronomy.

So far we have identified the following potential invited speakers (no
speakers have been contacted yet or have confirmed):

\begin{tabular}{l|l|l}
1a & Outreach                 & Pedro Russo (LU), Peter Barthel (RuG), Joeri van Leeuwen (UvA, Vici), Bas Haring, Ionica Smeets\\
1b & Instrumentation          & Paul Groot (RU), Peter Jonker (RU), Jelle Kaastra (SRON), Hiroki Akamatsu (SRON)\\
2a & Simulations and modeling & Amina Helmi (RuG, Vici + Spinoza), Adrian Hamers (IAS)\\
2b & Multi-messenger astronomy& Gijs Nelemans (RU, Vici), Heino Falcke (RU, Spinoza), Michiel van der Klis (UvA, Spinoza)\\
3a & Imaging                  & Marc Verheijen (RuG, Vici), Ewine van Dishoeck (LU, Spinoza)\\
3b & Spectroscopy             & Ignas Snellen (LU, Vici), Xander Tielens (LU, Spinoza)\\
4a & Theory                   & Sera Markoff (UvA, Vici), Erik Verlinde (UvA), Chris van den Broek (NIKHEF, Vici)\\
4b & Inteferometry            & Heino Falkce (RU, Spinoza)
\end{tabular}

We would like to give special emphasis to newly hired staff astronomers.
These include:
\begin{enumerate}
    \item Aurora Simionescu (LU)
    \item Yamilia Miguel (LU)
    \item Elisa Constantini (UvA)
    \item Elena Sellentin (LU)
\end{enumerate}
\end{document}
